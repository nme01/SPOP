\documentclass[11pt,a4paper]{article}

% basic packages
\usepackage{float}
\usepackage{fullpage}
\usepackage{polski}
\usepackage{amsmath}
\usepackage{graphicx}
\usepackage[utf8x]{inputenc}

% bibliography and links
\usepackage{url}
\usepackage{cite}
\def\UrlBreaks{\do\/\do-}
\usepackage[hidelinks]{hyperref}

% listings
\usepackage{listings}

\pagenumbering{gobble}

\begin{document}

\section*{Łamigłówka architekta}
\subsection*{Podział na pakiety}
Całość kodu została podzielona na~2 pakiety: \textit{UI} oraz \textit{Logic}.
Pierwszy z~pakietów odpowiada za~komunikację z~użytkownikiem (wczytanie
parametrów łamigłówki oraz prezentację rozwiązania). Pakiet \textit{Logic}
odpowiada za~znalezienie rozwiązania łamigłówki.

\subsection*{Komunikacja z~użytkownikiem}
Na początku działania programu wczytywane są~parametry łamigłówki z~pliku.
Za~wczytanie oraz przetworzenie danych z~pliku odpowiedzialna jest funkcja
\textit{loadPuzzle} z~modułu \newline \textit{UI.PuzzleLoader}.

Za~prezentację wyniku odpowiada funkcja \textit{prezentResult} z~modułu
\textit{UI.ResultPrezenter}. Funkcja ta~najpierw wypisuje w~konsoli
rozwiązanie, a~następnie zapisuje wynik do~wskazanego przez użytkownika
pliku. Za~wypisanie wyniku na~ekranie odpowiada funkcja \textit{printSolution}
z~modułu \textit{UI.ResultPrinter}.

\subsection*{Poszukiwanie rozwiązania}
Poszukiwanie rozwiązania przypomina przeszukiwanie wgłąb drzewa, w którym
węzły stanowią rozwiązania (częściowe lub całkowite, niekoniecznie poprawne).
Rozwiązaniem całkowitym jest takie rozwiązanie, w~którym każdy dom
ma~przydzielony swój zbiornik. Przy przyjęciu, że~przestrzeń rozwiązań
ma~postać drzewa, rozwiązaniami całkowitymi są liście drzewa.

Funkcją odpowiedzialną za~znalezienie rozwiązania jest funkcja
\textit{findSolution'} z~modułu \textit{Logic.SolutionFinder}. Ma~ona postać
rekurencyjną, gdzie warunkiem końcowym rekurencji jest znalezienie rozwiązania
całkowitego lub otrzymanie rozwiązania błędnego (częściowego lub całkowitego).
Funkcja \textit{findSolution'} jako jeden z~parametrów przyjmuje częściowe
rozwiązanie (lista przydziałów zbiornik$\rightarrow$dom). Dlatego moduł
\textit{Logic.SolutionFinder} udostępnia funkcję \textit{findSolution}
wywołującą funkcję \textit{findSolution'} z~pustą listą przydziałów.

Funkcja \textit{findSolution'} zawsze na~ początku sprawdza czy~otrzymane
rozwiązanie jest poprawne i~jeśli wszystkie domy mają przydzielone zbiorniki,
zwraca otrzymane rozwiązanie. Jeśli otrzymane rozwiązanie jest błędne, zwracana
jest lista pusta, co~oznacza porażkę w znaleziu rozwiązania.

Jeśli funkcja \textit{findSolution'} otrzyma niepustą listę domów, wybiera
jeden z domow, i~wywołuje dla niego funkcję \textit{movesIter}, której
wynik jest zwracany.

Funkcja \textit{movesIter} ma~za~zadanie znaleźć zbiornik dla zadanego domu
iterując po~możliwych pozycjach zbiornika względem domu,
a~następnie uruchomić \textit{findSolution'} dla pozostałych domów.
Jednym z~argumentow otrzymywanych przez \textit{movesIter} jest lista możliwych
pozycji zbiornika (przy wywołaniu najwyższego poziomu jest to~lista wszystkich
możliwych ruchów, niekoniecznie poprawnych), która jest zmniejszana wraz
z~iterowaniem po~kolejnych możliwych pozycjach zbiornika. Jeśli wywoływana
w~\textit{movesIter} funkcja \textit{findSolution'} zwróci rozwiązanie poprawne,
wówczas jest ono~zwracane. W~przeciwnym wypadku badane są kolejne dostępne
ruchy lub zwracana jest lista pusta (jeśli nie~ma~więcej możliwych ruchów).

Za~sprawdzanie poprawności rozwiązania odpowiada funkcja
\textit{hasSolutionErrors} z~modułu\newline \textit{Logic.Constraints}.
Sprawdza ona czy nie zostały złamane ograniczenia na liczbę zbiorników
w~wierszu/kolumnie, czy nowododany zbiornik nie koliduje z~innymi
obiektami i czy nie jest poza planszą. Dodatkowo w~module
\textit{Logic.Constraints} udostępniane są~funkcje \textit{updateRowConstr} oraz \textit{updateColumnConstr}
aktualizujące ograniczenia na liczbę zbiorników w~wierszach i~kolumnach
po~wykonaniu wybranego ruchu.

\end{document}
